\section*{Engine}
We decided to write our own DCR engine with Go. The reason for this was largely due to the fact
that we wanted to learn more about the intricacies of working with the DCR graphs to deepen our
understanding, and, of course, to challenge ourselves a bit.

\subsubsection*{XML Format}
First off, the input format; we created our own DCR graph input format, very much inspired by the
XML output you get from downloading a graph from \textit{dcrgraphs.net}. 
It is a simpler format where we start with an outer graph of type \textit{dcr} and move in from
there. Beneath, a simple overview:

\begin{itemize}
    \item[--] \textbf{dcr} 
    \begin{itemize}
        \item[--] \textbf{events} 
        \begin{itemize}
            \item[--] \textbf{event 1} 
        \end{itemize}	
        \item[--] \textbf{constraints} 
        \begin{itemize}
            \item[--] \textbf{conditions} 
            \item[--] \textbf{responses} 
            \begin{itemize}
                \item[--] \textbf{response 1} 
            \end{itemize}	
            \item[--] \textbf{corresponses} 
            \item[--] \textbf{excludes} 
            \item[--] \textbf{includes} 
            \item[--] \textbf{milestones} 
            \item[--] \textbf{spawns} 
        \end{itemize}	
        \item[--] \textbf{markings} 
        \begin{itemize}
            \item[--] \textbf{executed} 
            \item[--] \textbf{included} 
            \begin{itemize}
                \item[--] \textbf{event 1} 
            \end{itemize}	
            \item[--] \textbf{pending} 
        \end{itemize}	
    \end{itemize}	
\end{itemize}	

Now, we never implemented spawns, but they are within our format. One thing to note is that this
format does not support nesting. Looking at the actual XML provided should make entries such as the
response entry clear.

\subsection*{Engine overview}
The engine itself is fairly simple, consisting of three main files which make up the actual engine,
two files to implement some helper functions for data handling and XML and a single file to
execute. We will go over the three main files and omit the rest.

\textbf{DCR\_structs.go} is a short file where we declare the structs we want to use. Some these
structs are actually used to directly pick out information from the XML format we created
ourselves. 

\textbf{DCR\_interface.go} is where most of our engine is implemented. We define an interface
called DCR and the methods it should implement. In Go, this works a bit differently than in OOP
languages like C++ or Java; so, here, we just declare functions like so: \textit{func (struct)
Method name return type}. The struct here is the struct we want to add this method to, once
a struct has all methods of an interface, the interface is said to be implemented by the struct.
The implementation is very much our own interpretation of the code shown in the slides, so we won't
go into detail.

\textbf{DCR\_functions.go} holds the helper functions, which did not make sense to put onto the
struct of DCR\_graph. These functions are ones which act on something within the DCR\_graph struct,
but not the entire data structure. They are used throughout the interface implementations, as well as in the
\textbf{main.go} file to create the graphs from our XML files. 

\subsection*{Conformance Checker}
To check how a graph satisfies a trace we compile a set of all legal events
with the current marking of the graph. If the event is included in the set of legal
events the event is executed and the marking is updated. This process is repeated
every event in the trace. If any event is not in legal events is the trace not
satisfied. If every event is executed the set of pending events if check and if no
events are pending the trace the graph is concluded to satisfied the trace.

\subsubsection*{Tests}
We tested the result by hardcoding the patterns and check is we get the same
count of right and wrong traces. The specific of how each pattern is made can be
seen in the file test.go. With the testing we found a mishandel cases in the way
we handel pending relations. We found that at a event could be multiple times
ind the pending set of the marking and only one was removed when that event was
executed.
